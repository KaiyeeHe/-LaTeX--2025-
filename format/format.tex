\usepackage[CJKbookmarks,
colorlinks,
linkcolor=black,
anchorcolor=black,
citecolor=black,
urlcolor=black
]{hyperref}

\usepackage{shortvrb,makeidx}
\AtBeginDocument{
% APA citation Replace "and" with " & "
\DeclareDelimFormat{finalnamedelim}{%
  \ifnumgreater{\value{liststop}}{2}{\finalandcomma}{}%
  \addspace\&\space}
}

\usepackage[backend=biber, style=apa, sorting=nyt, sortlocale=zh__pinyin]{biblatex}

\DeclareSourcemap{
  \maps[datatype=bibtex]{
    \map{
      \step[fieldsource=langid, match=chinese, final]
      \step[fieldset=presort, fieldvalue={1}]
    }
    \map{
      \step[fieldsource=langid, match=english, final]
      \step[fieldset=presort, fieldvalue={2}]
    }
  }
}
% \usepackage[nosectionbib,unnumberedbib, tocbib, nourl]{apacite}
% \AtBeginDocument{
%     \renewcommand\bibname{参考文献}
%     \renewcommand\APACmonth{}
%     \renewcommand\BAvailFrom{}
%     \renewcommand\BRetrievedFrom{}
%     \renewcommand\BRetrieved{}
%     \renewcommand{\BBOP}{(}
%     \renewcommand{\BBCP}{)}
%     % \renewcommand{\APACrefauthstyle}{}
%     \renewcommand{\BRetrieved}[1]{} % Remove the urldate field
%     % \renewcommand\BRetrievedTo{}
% }

\usepackage{fancyhdr}
\usepackage{graphicx}
\usepackage{indentfirst,latexsym,colortbl,subfigure,clrscode}
\usepackage{algorithm}
\usepackage{algorithmic}
\usepackage{bm}                     % 处理数学公式中的黑斜体的宏包
\usepackage{amsmath}                % AMSLaTeX宏包 用来排出更加漂亮的公式
\usepackage{amssymb}                % AMSLaTeX宏包 用来排出更加漂亮的公式
\usepackage{mathrsfs}
\usepackage[subnum]{cases}
% \usepackage[numbers]{natbib} %sort&compress
\usepackage{geometry}
\usepackage{url}
\usepackage{times}
\usepackage{fontspec}
\usepackage{makecell,rotating,multirow,diagbox}
\usepackage{booktabs}
\usepackage{caption}
\usepackage{titletoc}
\usepackage{amsthm}
\usepackage{fourier} 
\usepackage{mathtools}

\urlstyle{same} %使用\url{}以开启链接的自动换行,该命令保证超链接文本字体样式与正文一致

\newcommand\cplus{\mathbin{\raisebox{-\height}{$+$}}}
\newcommand\contdots{\raisebox{-\height}{$\vphantom{+}\dotsm$}}
\newtheorem{hypothesis}{假设}[chapter]
\newif\ifnotanonymous\notanonymoustrue		%定义是否开启盲审,默认不开启
\newif\iftmlc

\def\cFracB#1#2{%
	\vcenter{\hbox{\strut$#1$\,\vrule}\hrule\hbox{\strut\vrule\,$#2$}}}
\makeindex
\pagestyle{fancy}

\renewcommand{\headrulewidth}{0.4pt}
\fancyfoot[RO,LE]{\thepage}
\fancyfoot[CO,CE]{}
\renewcommand{\algorithmicrequire}{\textbf{Input:}}
\renewcommand{\algorithmicensure}{\textbf{Output:}}
\newcommand{\loflabel}{图}
\newcommand{\lotlabel}{表}
\captionsetup[table]{labelsep=quad}
\captionsetup[figure]{labelsep=quad}


%                    根据自己正文需要做的一些定义                 %
%==================================================================%
\def\diag{{\rm diag}}
\def\rank{{\rm rank}}
\def\RR{{\cal R}}
\def\NN{{\cal N}}
\def\R{{\mathbb R}}
\def\C{{\mathbb C}}
\let\dis=\displaystyle

\def\p{\partial}
\def\f{\frac}
\def\mr{\mathrm}
\def\mb{\mathbf}
\def\mc{\mathcal}
\def\b{\begin}
\def\e{\end}

\newtheorem{thm1}{Theorem}[part]
\newtheorem{thm2}{Theorem}[section]
\newtheorem{thm3}{Theorem}[subsection]
\newtheorem{them}[thm2]{定理}
\newtheorem{theorem}[thm2]{定理}
\newtheorem{defn}[thm2]{定义}
\newtheorem{define}[thm2]{定义}
\newtheorem{ex}[thm2]{例}
\newtheorem{exs}[thm2]{例}
\newtheorem{example}[thm2]{例}
\newtheorem{prop}[thm2]{命题}
\newtheorem{lemma}[thm2]{引理}
\newtheorem{cor}[thm2]{推论}
\newtheorem{remark}[thm2]{注释}
\newtheorem{notation}[thm2]{记号}
\newtheorem{abbre}[thm2]{缩写}
% \newtheorem{algorithm}[thm2]{算法}
\newtheorem{problem}[thm2]{问题}
\newtheorem{Definition}{定义}[chapter]


\newcommand{\cyihao}{\fontsize{26pt}{39pt}\selectfont}			% 一号, 1.5 倍行距
\newcommand{\yihao}{\fontsize{26pt}{36pt}\selectfont}			% 一号, 1.4 倍行距
\newcommand{\erhao}{\fontsize{22pt}{28pt}\selectfont}			% 二号, 1.25倍行距
\newcommand{\xiaoer}{\fontsize{18pt}{18pt}\selectfont}          % 小二, 单倍行距
\newcommand{\sanhao}{\fontsize{16pt}{24pt}\selectfont}			% 三号, 1.5倍行距
\newcommand{\xiaosan}{\fontsize{15pt}{22pt}\selectfont}			% 小三, 1.5倍行距
\newcommand{\sihao}{\fontsize{14pt}{21pt}\selectfont}			% 四号, 1.5 倍行距
\newcommand{\banxiaosi}{\fontsize{13pt}{19.5pt}\selectfont}		% 半小四, 1.5倍行距
\newcommand{\xiaosi}{\fontsize{12pt}{18pt}\selectfont}			% 小四, 1.5倍行距
\newcommand{\dawuhao}{\fontsize{11pt}{11pt}\selectfont}			% 大五号, 单倍行距
\newcommand{\wuhao}{\fontsize{10.5pt}{15.75pt}\selectfont}		% 五号, 单倍行距

%============================ 可以自定义文字块 ================================%

\newcommand{\aaa}{Example}
\newcommand{\bbb}{\aaa \aaa \aaa}
\newcommand{\ccc}{\bbb \bbb \bbb \bbb \bbb
	
	\bbb \bbb \bbb \bbb \bbb }
\newcommand{\abc}{abcdefg1234567890}
\newcommand{\upabc}{ABCDEFGHIJK}
%%% ----------------------------------------------------------------------

\CTEXsetup[beforeskip = 0pt]{chapter}
\CTEXsetup[afterskip = 20pt]{chapter}


%============================= Layout 版型控制 ================================%
\setlength{\oddsidemargin}{0.57cm} 
\setlength{\evensidemargin}{\oddsidemargin}
 % \setlength{\parskip}{1em}
\voffset-6mm \textwidth=150mm \textheight=230mm \headwidth=150mm
%\rightmargin=35mm
%                                                                       %


%============================= 页面设置 ================================%
%-------------------- 定义页眉和页脚 使用fancyhdr 宏包 -----------------%
% 定义页眉与正文间双隔线
%my addition
\fancyhead{}
\fancyhead[C]{\bfseries 华东师范大学{\degreeCN}学位论文}
% \fancyhead[LO,RE]{\small\leftmark}
% % % % % % % % % % % % Explanation
% \fancyhead[L]{Left header on all pages}
% \fancyhead[CE]{Centered header on even pages}
% \fancyhead[RO]{Right header on odd pages}
% \fancyhead[L,R]{This appears on both left and right}
% \fancyhead[LE,RO]{Outer header}
%end myaddition

\newcommand{\makeheadrule}{%
	\makebox[0pt][l]{\rule[.7\baselineskip]{\headwidth}{0.4pt}}%
	\rule[0.85\baselineskip]{\headwidth}{0.4pt} \vskip-.8\baselineskip}
\makeatletter
\renewcommand{\headrule}{%
	{\if@fancyplain\let\headrulewidth\plainheadrulewidth\fi
		\makeheadrule}} \makeatother

\newcommand{\adots}{\mathinner{\mkern 2mu%
		\raisebox{0.1em}{.}\mkern 2mu\raisebox{0.4em}{.}%
		\mkernmu\raisebox{0.7em}{.}\mkern 1mu}}
\renewcommand{\contentsname}{\heiti \sanhao {目\quad 录}}
\renewcommand{\listfigurename}{\heiti \sanhao {图目录}}
\renewcommand{\listtablename}{\heiti \sanhao {表目录}}
\setmainfont{Times New Roman}
\dottedcontents{chapter}[1.5cm]{\sihao\heiti}{3.8em}{9.5pt}
\dottedcontents{section}[1.5cm]{\xiaosi\heiti}{2.8em}{9.5pt}
\DeclareRobustCommand\nobreakspace{\leavevmode\nobreak\ }




%============================= Bibitem control 参考文献================================% 
\def \auto {auto}
\ifx \refstyle \auto

\else

\usepackage{etoolbox} % Required, for availablility changing the latex env.
% \usepackage{hyperref} % Show a green-border box on each \cite label?

% Pre-define list variables
\makeatletter
\newcommand*{\lodbib@citeorder}{} % Catch normal entries
\newcommand*{\lodbib@notcited}{} % Catch entries that were not cited

% Macro in aux file
\def\citation{%
  \forcsvlist{\citation@i}}

\def\citation@i#1{%
  \ifinlist{#1}{\lodbib@citeorder}
    {}
    {\listxadd{\lodbib@citeorder}{#1}}}

\let\ltxorig@lbibitem\@lbibitem
\let\ltxorig@bibitem\@bibitem

% Save bibitems into lists
\def\@lbibitem[#1]#2#3{%
  \csdef{lodbib@savedlabel@#2}{#1}%
  \@bibitem{#2}{#3}}

\def\@bibitem#1#2{%
  \xifinlist{#1}{\lodbib@citeorder}
    {}
    {\listadd{\lodbib@notcited}{#1}}%
  \csdef{lodbib@savedentry@#1}{#2}}

% Re-define \thebibiography command.
\renewenvironment{thebibliography}[1]
     {\settowidth\labelwidth{\@biblabel{#1}}}
     {\def\@noitemerr
       {\@latex@warning{Empty `thebibliography' environment}}%
      \chapter*{参考文献} % chapter title
       \@mkboth{参考文献}{参考文献} % left page header / right page header
      \list{\@biblabel{\@arabic\c@enumiv}}%
           {\leftmargin\labelwidth\parsep=0pt
            \advance\leftmargin\labelsep
            \@openbib@code
            \usecounter{enumiv}%
            \let\p@enumiv\@empty
            \renewcommand\theenumiv{\@arabic\c@enumiv}}%
      \sloppy
      \clubpenalty4000
      \@clubpenalty \clubpenalty
      \widowpenalty4000%
      \sfcode`\.\@m
      \lodbib@biblistloop
      \endlist}
      
% Output the bib list.
\def\lodbib@biblistloop{%
  \forlistloop{\lodbib@bibitem}{\lodbib@citeorder}%
  \ifdefvoid{\lodbib@notcited}
    {}
    {\forlistloop{
    \lodbib@bibitem 
    }{
    %\lodbib@notcited % Whether listing uncited bib at the end of the references?
    }}}
    
% Modify the \cite num. in the book.
\def\lodbib@bibitem#1{%
  \ifcsundef{lodbib@savedlabel@#1}
    {\ltxorig@bibitem{#1}}
    {\ltxorig@lbibitem[\csuse{lodbib@savedlabel@#1}]{#1}}%
  \csuse{lodbib@savedentry@#1}}
  

\fi
 

%============== Convert all headers (incl. bib sec.) and footers into fancy style, with double underlines and book titles adding to the headers, pages num. adding to the footers.========%
\makeatletter
\let\ps@plain\ps@fancy

\usepackage{enumitem}
\setenumerate{itemsep=0pt,partopsep=0pt,parsep=\parskip,topsep=0pt}
\setitemize{itemsep=5pt,partopsep=0pt,parsep=\parskip,topsep=10pt,leftmargin=25pt,labelsep=10pt}
% \setdescription{itemsep=0pt,partopsep=0pt,parsep=\parskip,topsep=0pt}


%======================= TOC Settings =======================%
\titlecontents{chapter}[4.5em]{\sihao\heiti}%最前面到标题的缩进
              {\contentslabel{4em}}%序号和标题文字之间的缩进
              {\hspace{-4em}}%无序号标题往前缩进
              {\ \titlerule*[0.4pc]{$\cdot$}\contentspage}{} %引导线为"...1"
\titlecontents{section}[4.5em]{\heiti \xiaosi}%最前面到标题的缩进
              {\contentslabel{2.5em}}%序号和标题文字之间的缩进
              {}%
              {\ \titlerule*[0.4pc]{$\cdot$}\contentspage}{} %引导线为"...1"
\titlecontents{subsection}[7.5em]{}%最前面到标题的缩进
              {\contentslabel{2.9em}}%序号和标题文字之间的缩进
              {}%
              {\ \titlerule*[0.4pc]{$\cdot$}\contentspage}{} %引导线为"...1"
\titlecontents{figure}[0em]{}%最前面到标题的缩进
              {\figurename~\thecontentslabel \quad}%序号和标题文字之间的缩进
              {}%
              {\ \titlerule*[0.4pc]{$\cdot$}\contentspage}{} %引导线为"...1"
\titlecontents{table}[0em]{}%最前面到标题的缩进
              {表~\thecontentslabel \quad}%序号和标题文字之间的缩进
              {}%
              {\ \titlerule*[0.4pc]{$\cdot$}\contentspage}{} %引导线为"...1"

% Activate all \makeatletter script
\makeatother